%%%%%%%%%%%%%%%%%%%%%%%%%%%%%%%%%%%%%%%%
% Main document for our awesome monography.
%
% Webpage:
% https://github.com/subleleks/monography
%%%%%%%%%%%%%%%%%%%%%%%%%%%%%%%%%%%%%%%%

% Computer Science Bachelor's Degree
\documentclass[bacharelado]{unb-cic}

%%%%%%%%%%%%%%%%%%%%%%%%%%%%%%%%%%%%%%%%
% Imported Packages
%
% Besides cute colors, we're using the
% portuguese version.
%%%%%%%%%%%%%%%%%%%%%%%%%%%%%%%%%%%%%%%%

\usepackage[brazil,american]{babel}
\usepackage[T1]{fontenc}
\usepackage{indentfirst}
\usepackage{natbib}
\usepackage{xcolor,graphicx,url}
\usepackage[utf8]{inputenc}

%%%%%%%%%%%%%%%%%%%%%%%%%%%%%%%%%%%%%%%%
% Cores dos links
%%%%%%%%%%%%%%%%%%%%%%%%%%%%%%%%%%%%%%%%

% Veja o arquivos cores.tex se quiser ver que outras cores estão
% pré-definidas.  Utilizando o comando \hypersetup abaixo nós
% evitamos aquelas caixas vermelhas feias em volta dos links.

\input{cores}
\hypersetup{
  colorlinks=true,
  linkcolor=DarkOrange,
  citecolor=DarkOrange,
  filecolor=DarkOrange,
  urlcolor= DarkOrange
}


%%%%%%%%%%%%%%%%%%%%%%%%%%%%%%%%%%%%%%%%
% Monography Info
%%%%%%%%%%%%%%%%%%%%%%%%%%%%%%%%%%%%%%%%

\title{Título da monografia}

\orientador{\prof \dr Marcus Vinicius Lamar}{CIC/UnB}
\coorientador{\prof \dr Diego de Freitas Aranha}{CIC/Unicamp}
\coordenador{\prof \dr Coordenador}{CIC/UnB}
\diamesano{30}{março}{2014}

\membrobanca{\prof \dr Professor I}{CIC/UnB}
\membrobanca{\prof \dr Professor II}{CIC/UnB}

\autor{Alexandre Silva}{Dantas}
\coautor{Matheus Costa de Sousa Carvalho}{Pimenta}
\CDU{004.4}

\palavraschave{palvrachave1, palvrachave2, palvrachave3 }
\keywords{keyword1, keyword2, keyword3}

%%%%%%%%%%%%%%%%%%%%%%%%%%%%%%%%%%%%%%%%
% Texto
%%%%%%%%%%%%%%%%%%%%%%%%%%%%%%%%%%%%%%%%

\begin{document}
  \maketitle
  \pretextual

  \begin{dedicatoria}
  Dedico a....\textbf{mamãe}
  \end{dedicatoria}

  \begin{agradecimentos}
  Agradeço a....\textit{papai}
  \end{agradecimentos}

  \begin{resumo}
  A ciência...
  \end{resumo}

  \selectlanguage{american}
  \begin{abstract}
  The science...
  \end{abstract}
  \selectlanguage{brazil}

  \tableofcontents
  \listoffigures
  \listoftables

  \textual
  \chapter{Introdução}

% Objetivo - Construir um sistema computacional completo (hardware, software
%            básico, aplicação) para máquinas de instrução única (OISC)
%

\section{Conceitos Básicos}

% Definir
% OISC
% RISC

\section{Linguagem de Montagem Subleq}
\label{sec:subleq}

Como visto em \cite{subleq}, a instrução \textit{Subleq} é muito bonita.

% Pontos positivos
%
% Pontos negativos
% * Impacto na velocidade de execução


  
\chapter{Referencial Teórico}

\section{Arquitetura de Processadores}

CPU

\section{Controladores de Dispositivos Externos}

CPU

\section{Sistemas Operacionais}

CPU

\section{Carregadores}

CPU

\section{Ligadores}

CPU

\section{Montadores}

CPU

\section{Compiladores}

CPU

\section{Criptografia}

CPU

  % ...

  \postextual
  \bibliographystyle{plain}
  \bibliography{bibliografia}

\end{document}
